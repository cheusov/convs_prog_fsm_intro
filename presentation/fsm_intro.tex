%%%begin-myprojects
\documentclass[hyperref={colorlinks=true},xcolor=table]{beamer}

\usepackage{fancyvrb,relsize}
\usepackage{graphicx}

\usetheme{Malmoe}
%\usetheme{Berlin}
%\usetheme{Warsaw} % Boadilla 

\setbeamertemplate{navigation symbols}{}
\setbeamertemplate{page number in head/foot}[totalframenumber]

%\usetheme{Boadilla}
%\usetheme{CambridgeUS}
%\usetheme{Malmoe}
%\usetheme{Singapore}
%\usetheme{boxes}

%\usecolortheme{crane}
%\usecolortheme{dove}
\usepackage[linesnumbered,ruled]{algorithm2e}
\usecolortheme{seagull} % very cool with \usetheme{default}
%\usecolortheme{lily} % very cool with \usetheme{default}
%\usefonttheme{professionalfonts}
%\useinnertheme{rectangles}

\mode<presentation>
\title{Introduction to Finite State Machines. Part I}
\author{Aleksey Cheusov \\ \texttt{vle@gmx.net} \\   t.me/convs\_prog}
\date{Minsk/Belarus, Kutaisi/Georgia\\
  September 2024\\
}

\begin{document}

%%%%%%%%%%%%%%%%%%%%%%%%%%%%%%%%%%%%%%%%%%%%%%%%%%%%%%%%%%%%%%%%%%%%%%

\newenvironment{Code}[1]%
              {\Verbatim[label=\bf{#1},frame=single,%
                  fontsize=\small,%
                  commandchars=\\\{\}]}%
              {\endVerbatim}

%\newenvironment{Code}[1]%
%               {\semiverbatim[]}%
%               {\endsemiverbatim}

\newenvironment{CodeNoLabel}%
               {\Verbatim[frame=single,%
                   fontsize=\small,%
                   commandchars=\\\{\}]}%
               {\endVerbatim}
\newenvironment{CodeNoLabelTiny}%
               {\Verbatim[frame=single,%
                   fontsize=\tiny,%
                   commandchars=\\\{\}]}%
               {\endVerbatim}
\newenvironment{CodeNoLabelSmallest}%
               {\Verbatim[frame=single,%
                   fontsize=\footnotesize,%
                   commandchars=\\\{\}]}%
               {\endVerbatim}
\newenvironment{CodeLarge}%
               {\Verbatim[frame=single,%
                   fontsize=\large,%
                   commandchars=\\\{\}]}%
               {\endVerbatim}

%\newcommand{\prompt}[1]{\textcolor{blue}{#1}}
%\newcommand{\prompt}[1]{\textbf{#1}\textnormal{}}
\newcommand{\prompt}[1]{{\bf{#1}}}
%\newcommand{\h}[1]{\textbf{#1}}
%\newcommand{\h}[1]{\bf{#1}\textnormal{}}
\newcommand{\h}[1]{{\bf{#1}}}
\newcommand{\URL}[1]{\textbf{#1}}
\newcommand{\AutohellFile}[1]{\textcolor{red}{#1}}
\newcommand{\MKCfile}[1]{\textcolor{green}{#1}}
\newcommand{\ModuleName}[1]{\textbf{#1}\textnormal{}}
\newcommand{\ProgName}[1]{\textbf{#1}\textnormal{}}
\newcommand{\ProjectName}[1]{\textbf{#1}\textnormal{}}
\newcommand{\PackageName}[1]{\textbf{#1}\textnormal{}}
\newcommand{\MKC}[1]{\large\textsf{#1}\textnormal{}\normalsize}

%%%%%%%%%%%%%%%%%%%%%%%%%%%%%%%%%%%%%%%%%%%%%%%%%%%%%%%%%%%%%%%%%%%%%%
%% \begin{frame}
%%   \frametitle{qqq}
%%   \begin{code}{files in the directory}
%%     bla bla bla
%%   \end{code}
%% \end{frame}

%%%end-myprojects

%%%%%%%%%%%%%%%%%%%%%%%%%%%%%%%%%%%%%%%%%%%%%%%%%%%%%%%%%%%%%%%%%%%%%%
\begin{frame}
  \titlepage
\end{frame}

%%%%%%%%%%%%%%%%%%%%%%%%%%%%%%%%%%%%%%%%%%%%%%%%%%%%%%%%%%%%%%%%%%%%%%
\begin{frame}
  \frametitle{Yet another presentation about Finite State Machines?}
  Goals
  \begin{itemize}
  \item Short introduction to Finite State Machines
  \item Open source educational tool \textbf{my\_grep} for learning
  \item Real-life benchmarks with a lot of avalable regexps implementations
  \end{itemize}
  Useful side-effect ;-)
  \begin{itemize}
  \item Newer -- does not mean better
  \item More popular -- does not mean better
  \item Actively developped -- does not mean better
  \item Big community -- does not mean better
  \item Richer functionality -- does not mean better
  \item Packaged in every Linux/BSD system -- does not mean better
  \end{itemize}
\end{frame}

%%%%%%%%%%%%%%%%%%%%%%%%%%%%%%%%%%%%%%%%%%%%%%%%%%%%%%%%%%%%%%%%%%%%%%
\begin{frame}
  \frametitle{Plan}
  \begin{itemize}
  \item Formal definition of Finite State Automaton (Non-deterministic
    Finite Automaton or NFA)
  \item Examples of Finite State Automata
  \item Formal definition of Deterministic Finite Automaton (DFA)
  \item Examples of Deterministic Finite Automata
    % division by 2
    % division by 3
    % accepting openvpn within a string (Knuth�Morris�Pratt and
    %  Aho-Corasick algorithms)
    % accepting floating point number
  \item Formal definition of regular language
  \item Regular expressions and glob patterns
    % equivalence of regular language and regular expression, regexp
    % libraries are often NOT a regular ex[pression in methematical sense
  \item Algorithm of match with the help of DFA
  \item my\_grep: grep-like tool for matching using glob pattern
    \begin{itemize}
    \item Convertion of glob pattern to finite state automaton
      % files for testing, input{0,1,2,3,4,5}
    \item Benchmarks for POSIX regexp implementations (libc, libtre, libpcre2,
      liboniguruma, libuxre, rxspencer), Google re2, Yandex PIRE, C++
      std::regex, GNU grep, Perl, Ruby, GNU awk,
      mawk, nbawk (nawk by Brian Kernighan patched by NetBSD developers)
    \end{itemize}
  \item Conslusions
  \end{itemize}
\end{frame}

%%%%%%%%%%%%%%%%%%%%%%%%%%%%%%%%%%%%%%%%%%%%%%%%%%%%%%%%%%%%%%%%%%%%%%
\begin{frame}
  \frametitle{Finite State Automaton (FSA), also known as
    Non-deterministic Finite Automaton (NFA)}
  A finite state automaton is a 5-tuple $<\Sigma,S,Q,F,\delta>$. Sometimes it is called ``acceptor''.
  \begin{itemize}
  \item $\Sigma$ is the alphabet, a finite non-empty set of symbols.
  \item $S$ is a finite non-empty set of states.
  \item $Q$ is the set of initial states, $Q \subseteq S$.
  \item $F$ is the set of final states, $F \subseteq S$.
  \item $\delta$ is the transition relation: $\delta \subseteq S \times \Sigma \times S$
    (or, alternatively, $\delta: S \times \Sigma \rightarrow 2^S$)
  \end{itemize}
  Example: $<\{a,b\},\{s1,s2,s3\},\{s1\},\{s3\},\delta>$ where\\
  $\delta$ = $\{(s1, a, s3), (s1, a, s2), (s2, b, s1)\}$
  \begin{figure}
    \includegraphics[height=0.3\textheight]{fsa_example.eps}
  \end{figure}
\end{frame}

%%%%%%%%%%%%%%%%%%%%%%%%%%%%%%%%%%%%%%%%%%%%%%%%%%%%%%%%%%%%%%%%%%%%%%
\begin{frame}
  \frametitle{Language of FSA}
  The language formed by the FSA consists
  of all distinct strings that can be accepted by FSA, i.e. sequences of symbols 
  that start in any initial state and ends in a final state. $L(fsa) = \{(a b)^n a | n \ge 0\}$
  \begin{figure}
    \includegraphics[keepaspectratio=false]{fsa_example.eps}
  \end{figure}
  \textbf{Note:} FSAs are equivalent if their languages are equal.
\end{frame}

%%%%%%%%%%%%%%%%%%%%%%%%%%%%%%%%%%%%%%%%%%%%%%%%%%%%%%%%%%%%%%%%%%%%%%
\begin{frame}
  \frametitle{Example. FSA ``devided by 2''}
  FSA: $<\{0,1\},\{r0,r1\},\{r0\},\{r0\},\delta>$ where\\
  $\delta$ = $\{(r0, 0, r0), (r0, 1, r1), (r1, 0, r0), (r1, 1, r1)\}$\\
  Most significant bits come first.
  \begin{figure}
    \includegraphics[keepaspectratio=true]{fsa_divided_by_2.eps}
  \end{figure}
\end{frame}

%%%%%%%%%%%%%%%%%%%%%%%%%%%%%%%%%%%%%%%%%%%%%%%%%%%%%%%%%%%%%%%%%%%%%%
\smaller
\begin{frame}
  \frametitle{Example. FSA ``devided by 3''}
  FSA: $<\{0,1\},\{r0,r1, r2\},\{r0\},\{r0\},\delta>$ where\\
  $\delta$ = $\{(r0, 0, r0), (r0, 1, r1), (r1, 1, r0), (r1, 0, r2), (r2, 1, r2), (r2, 0, r1)\}$\\
  Most significant bits come first.
  \begin{figure}
    \includegraphics[keepaspectratio=true]{fsa_divided_by_3.eps}
  \end{figure}
\end{frame}
\normalsize

%%%%%%%%%%%%%%%%%%%%%%%%%%%%%%%%%%%%%%%%%%%%%%%%%%%%%%%%%%%%%%%%%%%%%%
\smaller
\begin{frame}
  \frametitle{Example. FSA for finding occurrence of the substring ``openvpn'' in the string}
  FSA: $<\{o,p,e,n,v,?\},\{0,1,2,3,4,5,6,7\},\{0\},\{7\},\delta>$
  \begin{figure}
    \includegraphics[width=1.0\textwidth, keepaspectratio=true]{fsa_openvpn.eps}
  \end{figure}
\end{frame}
\normalsize

%%%%%%%%%%%%%%%%%%%%%%%%%%%%%%%%%%%%%%%%%%%%%%%%%%%%%%%%%%%%%%%%%%%%%%
\smaller
\begin{frame}
  \frametitle{Example. FSA for matching floating point number}
  FSA: $<\{+,-,.,[0-9]\},\{s0,s1,s2,s3,s4,s5\},\{s0,s1\},\{s2,s3\},\delta>$
  \begin{figure}
    \includegraphics[width=0.9\textwidth, keepaspectratio=true]{fsa_number.eps}
  \end{figure}
\end{frame}
\normalsize

%%%%%%%%%%%%%%%%%%%%%%%%%%%%%%%%%%%%%%%%%%%%%%%%%%%%%%%%%%%%%%%%%%%%%%
\begin{frame}[fragile]
  \frametitle{Example. FSA for software design and testing}
  \textbf{State diagram for C++ class for binary operation:}
  \begin{block}{}
      \begin{CodeNoLabelTiny}
template <typename TArg, typename TRes> class IBinaryOperation \{
    virtual void set\_a(TArg a) = 0;
    virtual void set\_b(TArg b) = 0;
    virtual void process() = 0;
    virtual TRes get\_result() = 0;
    virtual void clear() = 0;
\}
      \end{CodeNoLabelTiny}
  \end{block}
%  \begin{block}
    \begin{figure}
      \includegraphics[width=0.71\textwidth, height=0.4\textheight, keepaspectratio=false]{bin_operation.eps}
    \end{figure}
%  \end{block}
\end{frame}

%%%%%%%%%%%%%%%%%%%%%%%%%%%%%%%%%%%%%%%%%%%%%%%%%%%%%%%%%%%%%%%%%%%%%%
\smaller
\begin{frame}
  \frametitle{Example. FSA for software design and testing}
  \begin{table}
    \begin{tabular}{l | c | c | c }
      \rowcolor{lightgray}
      & set\_a & set\_b & process \\
      \hline
      \cellcolor{lightgray} initial & A\_is\_ready & B\_is\_ready & Error! \\ 
      \cellcolor{lightgray} A\_is\_ready & A\_is\_ready & A\_B\_are\_ready & Error! \\
      \cellcolor{lightgray} B\_is\_ready & A\_B\_are\_ready & B\_is\_ready & Error! \\
      \cellcolor{lightgray} A\_B\_are\_ready & A\_B\_are\_ready & A\_B\_are\_ready & result\_is\_ready\\
      \cellcolor{lightgray} result\_is\_ready & A\_B\_are\_ready & A\_B\_are\_ready & Error!\\
    \end{tabular}
    \caption{Transition table, part 1}
  \end{table}
  \begin{table}
    \begin{tabular}{l | c | c |}
      \rowcolor{lightgray}
       &  get\_result & clear \\
      \hline
      \cellcolor{lightgray} initial & Error! & initial \\ 
      \cellcolor{lightgray} A\_is\_ready & Error! & initial\\
      \cellcolor{lightgray} B\_is\_ready & Error! & initial\\
      \cellcolor{lightgray} A\_B\_are\_ready & Error! & initial\\
      \cellcolor{lightgray} result\_is\_ready & result\_is\_ready & initial \\
    \end{tabular}
    \caption{Transition table, part 2}
  \end{table}
\end{frame}
\normalsize

%%%%%%%%%%%%%%%%%%%%%%%%%%%%%%%%%%%%%%%%%%%%%%%%%%%%%%%%%%%%%%%%%%%%%%
\smaller
\begin{frame}
  \frametitle{Deterministic Finite Automaton (DFA)}
  A deterministic finite state automaton is a 5-tuple $<\Sigma,S,q,F,\delta>$.
  \begin{itemize}
  \item $\Sigma$ is the alphabet, a finite non-empty set of symbols.
  \item $S$ is a finite non-empty set of states.
  \item $q$ is the initial state, $q \in S$.
  \item $F$ is the set of final states, $F \subseteq S$.
  \item $\delta$ is the state-transition function: $\delta: S \times \Sigma \rightarrow S$
  \end{itemize}
  \textbf{Theorem:} NFA and DFA are equivalent formalisms. That is,
  for any given NFA it is possible to construct equivalent DFA and
  vice versa.
  \break
  \textbf{Theorem:} DFA can be exponentially larger than equivalent NFA.
  \break
  \textbf{Note:} There is algorithm that converts NFA to DFA.
  \break
  \textbf{Note:} There is algorithm that converts NFA to Minimal DFA.
  \break
  \textbf{Note:} There are algorithms that convert DFA to Minimal DFA.
  \break
  \textbf{Theorem:} For any finite state automaton, there is a single Minimal
  DFA equivalent to the given one.
\end{frame}
\normalsize

%%%%%%%%%%%%%%%%%%%%%%%%%%%%%%%%%%%%%%%%%%%%%%%%%%%%%%%%%%%%%%%%%%%%%%
\begin{frame}
  \frametitle{Example. NFA and equivalent DFA}
  NFA
  \begin{figure}
    \includegraphics[width=0.40\textwidth]{fsa_example.eps}
  \end{figure}
  Equivalent minimal DFA
  \begin{figure}
    \includegraphics[width=0.30\textwidth]{dfa_example.eps}
  \end{figure}
  Language: $\{(a b)^n a | n \ge 0\}$
\end{frame}

%%%%%%%%%%%%%%%%%%%%%%%%%%%%%%%%%%%%%%%%%%%%%%%%%%%%%%%%%%%%%%%%%%%%%%
\begin{frame}
  \frametitle{Example. DFA and equivalent minimal DFA}
  DFA
  \begin{figure}
    \includegraphics[width=0.40\textwidth]{dfa_example1.eps}
  \end{figure}
  Equivalent minimal DFA
  \begin{figure}
    \includegraphics[width=0.30\textwidth]{mindfa_example1.eps}
  \end{figure}
  Language: $\{(a|b)^n | n \ge 1\}$
\end{frame}

%%%%%%%%%%%%%%%%%%%%%%%%%%%%%%%%%%%%%%%%%%%%%%%%%%%%%%%%%%%%%%%%%%%%%%
\smaller
\begin{frame}
  \frametitle{Regular language}
  The collection of regular languages over an alphabet $\Sigma$ is defined recursively as follows:
  \begin{itemize}
  \item The empty language $\emptyset$, and the empty string language $\{\epsilon\}$ are regular languages.
  \item For each $a \in \Sigma$, the singleton language $\{a\}$ is a regular language.
  \item If $A$ and $B$ are regular languages, then $A \cup B$ (union), $A \bullet B$ (concatenation), and $A\ast$ (Kleene star) are regular languages.
  \item No other languages over $\Sigma$ are regular.
  \end{itemize}
  Above formalism gives us so called ``\textbf{regular expressions}'', for
  example,\\
  \center{\texttt{\^{}Sep 23.*(openvpn|sshd)}}
%  ``\^{}Sep 23.* (openvpn \vert sshd)''.
  \break\break
  \textbf{Theorem:} Regular Language and Finite State Automaton are equivalent
  formalisms. That is, for each regular language the equivalent FSA
  exists and vise versa.
  \break
  \textbf{Note:} Language that supports
  back-references is not regular and therefore matcher cannot be
  \textbf{efficiently} implemented
  with the help of FSA. %That's why loosers work so badly.
%  \textbf{Theorem:} Regular languages are closed under concatenation, union, kleene star,
%  intersection and complementation.
\end{frame}
\normalsize

%%%%%%%%%%%%%%%%%%%%%%%%%%%%%%%%%%%%%%%%%%%%%%%%%%%%%%%%%%%%%%%%%%%%%%
\begin{frame}
  \frametitle{Glob pattern}
  Glob pattern is a way we specify the regular language:
  \begin{itemize}
  \item The wildcard * stands for ``any string of any length including empty''
  \item The wildcard ? stands for ``any character''
  \item All other characters mean itself
  \end{itemize}
  \break
  I implemented my own tool \textbf{my\_grep} for matching
  using \textbf{glob} patterns.
  \break
  \break
  \textit{usage: my\_grep PATTERN FILE}
  \break
  \break
  \textbf{my\_grep} searches for PATTERN in FILE,
  and prints each line that matches a PATTERN. PATTERN is a glob
  pattern as described above. If FILE is -, then stdin is used as input.
\end{frame}

%%%%%%%%%%%%%%%%%%%%%%%%%%%%%%%%%%%%%%%%%%%%%%%%%%%%%%%%%%%%%%%%%%%%%%
\tiny
\begin{frame}
  \frametitle{Algorithm of building FSA from glob pattern}
  \begin{algorithm}[H]
    \DontPrintSemicolon
    \SetKwInOut{Input}{input}
    \SetKwInOut{Output}{output}
    \Input{GlobPattern = $[c_1, c_2 \dots c_n]$}
    \Output{FSA = $<\Sigma,S,Q,F,\delta>$}
    \SetAlgoLined
%    \SetAlgoNoEnd
    $\Sigma \leftarrow \{?\}$\;
    $\delta \leftarrow \emptyset$\;
    \For{$i \in [1,n]$}{
      \uIf{$c_i \ne *$   and   $c_i \ne ?$}{
        $\Sigma \leftarrow \Sigma \cap {c_i}$
      }
    }
    $state \leftarrow 0$\;
    \For{$i \in [1,n]$}{
      \uIf{$c_i$ = *}{
        \For{$s \in \Sigma$}{$\delta \leftarrow \delta \cap (state, s, state)$ }
      }\uElseIf{$c_i$ = ?}{
        \For{$s \in \Sigma$}{
          $\delta \leftarrow \delta \cap (state, s, state + 1)$\;
        }
        $state \leftarrow state + 1$\;
      }\uElse{
        $\delta \leftarrow \delta \cap (state, c_i, state + 1)$\;
        $state \leftarrow state + 1$\;
      }
    }
    \textbf{return} $<\Sigma, \{ x \in \mathbb{Z} | 0 \leq x \leq state \},
    \{0\}, \{state\}, \delta>$
    \caption{Convert glob pattern to FSA. \textbf{Complexity: $O(n)$}}
  \end{algorithm}
\end{frame}
\normalsize

%%%%%%%%%%%%%%%%%%%%%%%%%%%%%%%%%%%%%%%%%%%%%%%%%%%%%%%%%%%%%%%%%%%%%%
\begin{frame}
  \frametitle{ Example. FSA for glob pattern Sep 2?*sshd }
  Glob pattern:
  Sep 2?*sshd
  \break\break
  FSA:
  \begin{figure}
    \includegraphics[height=0.50\textheight]{fsa_sep_23_star_openvpn.eps}
  \end{figure}
\end{frame}

%%%%%%%%%%%%%%%%%%%%%%%%%%%%%%%%%%%%%%%%%%%%%%%%%%%%%%%%%%%%%%%%%%%%%%
\begin{frame}
  \frametitle{Algorithm of match with the help of DFA}
  \begin{algorithm}[H]
    \DontPrintSemicolon
    \SetKwInOut{Input}{input}
    \SetKwInOut{Output}{output}
    \Input{DFA = $<\Sigma,S,q,F,\delta>, Text=[t_1, t_2 \dots t_n], t_i \in \Sigma$}
    \Output{$true$ \textbf{or} $false$}
    \SetAlgoLined
%    \SetAlgoNoEnd
    $state \leftarrow q$\;
    \For{$i \in [1,n]$}{
      \uIf{$\delta$ is defined on $(state, t_i)$}{
        $state \leftarrow \delta(state, t_i)$\;
      }\Else{
        \textbf{return} false\;
      }
    }
    \textbf{return} $(state \in F)$\;
    \caption{Match with a help of DFA. \textbf{Complexity: $O(n)$}}
  \end{algorithm}
\end{frame}

%%%%%%%%%%%%%%%%%%%%%%%%%%%%%%%%%%%%%%%%%%%%%%%%%%%%%%%%%%%%%%%%%%%%%%
\begin{frame}[fragile]
  \frametitle{Utility my\_grep. C++ class fast\_dfa}
  If $\delta$ is not defined on $(state, iw)$,\break
  m\_arcs[state * m\_iw\_count + iw] == -1
  \begin{block}{}
    \begin{CodeNoLabelTiny}
class fast_dfa \{
private:
   unsigned m_iw_count;
   unsigned *m_arcs = nullptr;
   bool *m_finite_states = nullptr;

public:
   inline
   int get_arc(int state, unsigned iw) const \{
      return m_arcs[state * m_iw_count + iw];
   \}

   inline
   bool is_finite_state(int state) const \{
      return m_finite_states[state];
   \}

   ...
\};
    \end{CodeNoLabelTiny}
  \end{block}
\end{frame}

%%%%%%%%%%%%%%%%%%%%%%%%%%%%%%%%%%%%%%%%%%%%%%%%%%%%%%%%%%%%%%%%%%%%%%
\begin{frame}[fragile]
  \frametitle{Utility my\_grep. C++ class dfa\_matcher}
  \begin{block}{}
    \begin{CodeNoLabel}
class dfa_matcher \{
private:
   // DFA
   fast_dfa m_fast_dfa;

   // map symbols used in regexp to 1, 2 etc.
   // and all others to 0
   unsigned *m_iw_map = nullptr;
   unsigned m_iw_map_size = 0;

public:
   // return non-zero value if match succeeded
   // and 0 otherwise
   int match(const char *buffer, size_t buffer_size);
   ...
\}
    \end{CodeNoLabel}
  \end{block}
\end{frame}

%%%%%%%%%%%%%%%%%%%%%%%%%%%%%%%%%%%%%%%%%%%%%%%%%%%%%%%%%%%%%%%%%%%%%%
\begin{frame}[fragile]
  \frametitle{Utility my\_grep. Method dfa\_matcher::match}
  \begin{block}{}
    \begin{CodeNoLabel}
   int glob_match::match(
      const char *buffer, size_t buffer_size)
   \{
      unsigned iw = 0;
      int state = 0; // 0 is a initial state of DFA

      for (size_t pos = 0; pos < buffer_size; ++pos) \{
         iw = (unsigned) (unsigned char) buffer[pos];
         iw = m_iw_map[iw];
         state = m_fast_dfa.get_arc(state, iw);
         if (state < 0)
            return 0;
      \}

      return m_fast_dfa.is_finite_state(state);
   \}
    \end{CodeNoLabel}
  \end{block}
\end{frame}

%%%%%%%%%%%%%%%%%%%%%%%%%%%%%%%%%%%%%%%%%%%%%%%%%%%%%%%%%%%%%%%%%%%%%%
\begin{frame}[fragile]
  \frametitle{Utility my\_grep. ``Completely finite'' states}
  Glob pattern: \textbf{ab*}
  \break
  $\Sigma = \{a,b,?\}$
  \break
  \begin{figure}
    \includegraphics[height=0.35\textheight]{dfa_abstar.eps}
  \end{figure}
  As soon as we reach state \textbf{s3} we can stop scanning and return
  ``matched''. Thus, we reduce the number of input symbols to scan.
  Let's call states like \textbf{s3} ``completely finite''.
\end{frame}

%%%%%%%%%%%%%%%%%%%%%%%%%%%%%%%%%%%%%%%%%%%%%%%%%%%%%%%%%%%%%%%%%%%%%%
\begin{frame}[fragile]
  \frametitle{Utility my\_grep. ``Completely finite'' states}
  In order to optimize ``completely finite'' states, let's make
  \textbf{m\_fast\_dfa.get\_arc(state, iw)} equal to \textbf{-2}
  when state is ``completely finite'' and \textbf{-1} when $\delta$ is not
  defined on $(state, iw)$.

  \begin{block}{}
    \begin{CodeNoLabelTiny}
   int glob_match::match(const char *buffer, size_t buffer_size)
   \{
      unsigned iw = 0;
      int state = 0; // 0 is a initial state of DFA

      for (size_t pos = 0; pos < buffer_size; ++pos) \{
         iw = (unsigned) (unsigned char) buffer[pos];
         iw = m_iw_map[iw];
         state = m_fast_dfa.get_arc(state, iw);
         if (state < 0)
            // unoptimized version: return 0
            return state + 1; // -2 -> -1(matched);   -1 -> 0(not matched)
      \}

      return m_fast_dfa.is_finite_state(state);
   \}
    \end{CodeNoLabelTiny}
  \end{block}
\end{frame}

%%%%%%%%%%%%%%%%%%%%%%%%%%%%%%%%%%%%%%%%%%%%%%%%%%%%%%%%%%%%%%%%%%%%%%
\begin{frame}[fragile]
  \frametitle{Utility my\_grep. Optimization of finite states}
  We can renumber states in a way that each finite state is greater
  than non-finite one. In this case, we can remove m\_finite\_states array.

  \begin{block}{}
    \begin{CodeNoLabelTiny}
   class fast_dfa \{
   private:
      ...
      unsigned *m_arcs = nullptr;
      unsigned m_initial_state; // new member
      unsigned m_first_finite_state; // new member
      // bool *m_finite_states = nullptr; // we don't need this array any longer
   public:
      inline bool is_finite_state(int state) const noexcept \{
         return state >= m_first_finite_state; // operation >= instead of accessing memory!
      \}
      inline unsigned get_initial_state() const noexcept \{
         return m_initial_state;
      \}
   \}
    \end{CodeNoLabelTiny}
  \end{block}
\end{frame}

%%%%%%%%%%%%%%%%%%%%%%%%%%%%%%%%%%%%%%%%%%%%%%%%%%%%%%%%%%%%%%%%%%%%%%
\begin{frame}[fragile]
  \frametitle{Utility my\_grep. Optimization of finite states}

  Optimized version:
  \begin{block}{}
    \begin{CodeNoLabelTiny}
   int glob_match::match(const char *buffer, size_t buffer_size)
   \{
      unsigned iw = 0;
      // unoptimized version: int state = 0;
      int state = m_fast_dfa.get_initial_state();

      for (size_t pos = 0; pos < buffer_size; ++pos) \{
         iw = (unsigned) (unsigned char) buffer[pos];
         iw = m_iw_map[iw];
         state = m_fast_dfa.get_arc(state, iw);
         if (state < 0)
            return state + 1;
      \}

      return m_fast_dfa.is_finite_state(state);
   \}
    \end{CodeNoLabelTiny}
  \end{block}
\end{frame}

%%%%%%%%%%%%%%%%%%%%%%%%%%%%%%%%%%%%%%%%%%%%%%%%%%%%%%%%%%%%%%%%%%%%%%
\begin{frame}[fragile]
  \frametitle{Benchmark. List of libraries and utilities we test}
  Hardware: AMD Ryzen 7 3700X.\break\break
  Tested libraries and tools:
  \footnotesize
  \begin{itemize}
  \item Based on POSIX regcomp/regexec API
    \begin{itemize}
    \item GNU libc 2.32
    \item NetBSD libc 10.99.10
    \item librxspencer 3.8.4 by Henry Spencer
    \item libtre 0.8.0 by Ville Laurikari
    \item libpcre2 10.42
    \item liboniguruma 6.9.9 by K.Kosako
    \item libuxre 070715 (heirloom, opensourced by Caldera in 2001)
    \end{itemize}
  \item Google re2 20220601
  \item Yandex PIRE 0.0.6 by Dmitry Prokoptsev, Alexander Gololobov et al
  \item GNU grep 3.6.0.18.70517
  \item GNU awk 5.1.0 by Aaron Robins
  \item mawk 1.3.4.20200120 by Mike Brennan and Thomas E. Dickey
  \item NetBSD awk (nbase, nawk by Brian Kernighan) 9.3
  \item perl 5.34.0
  \item ruby 3.1.4p223
  \end{itemize}
  \normalsize
\end{frame}

%%%%%%%%%%%%%%%%%%%%%%%%%%%%%%%%%%%%%%%%%%%%%%%%%%%%%%%%%%%%%%%%%%%%%%
\begin{frame}[fragile]
  \frametitle{Benchmark. Important notes}
  \begin{itemize}
  \item This is \textbf{not} a REAL benchmark
    \begin{itemize}
    \item The only tested mode is \textbf{matched/unmatched}. No substring
      search!
    \item Input text files are artificial (based on system-wide
      \textbf{/usr/share/dict/words})
    \item There is only \textbf{one} tested regular expression, and
      it's trivial (glob: \textbf{*a*b*c*d*}, regexp: \textbf{a.*b.*c.*d}).
    \end{itemize}
  \item My only goal is to demonstrate the importance of algorithm
    \textbf{complexity} and nothing else.
  \item All benchmark tools work like POSIX grep utility. They take
    pattern and file as command line arguments and output matched
    lines to stdout.
  \end{itemize}
\end{frame}

%%%%%%%%%%%%%%%%%%%%%%%%%%%%%%%%%%%%%%%%%%%%%%%%%%%%%%%%%%%%%%%%%%%%%%
\begin{frame}[fragile]
  \frametitle{Benchmark. POSIX regcomp/regexec tools}
  Regexp compilation
  \begin{block}{}
    \begin{CodeNoLabelTiny}
      const char *pattern;
      ...
      static regex\_t regex;
      int ret = regcomp(&regex, pattern, REG\_EXTENDED | REG\_NOSUB);
      ...
    \end{CodeNoLabelTiny}
  \end{block}
  \break\break
  String match
  \begin{block}{}
    \begin{CodeNoLabelTiny}
      const char *line;
      ...
      int ret = regexec(&regex, line, 0, NULL, 0);
      ...
    \end{CodeNoLabelTiny}
  \end{block}
\end{frame}

%%%%%%%%%%%%%%%%%%%%%%%%%%%%%%%%%%%%%%%%%%%%%%%%%%%%%%%%%%%%%%%%%%%%%%
\begin{frame}[fragile]
  \frametitle{Benchmark. AWK, Perl and Ryby}
  grep and GNU grep
  \begin{block}{}
    \begin{CodeNoLabelTiny}
      grep 'pattern'
    \end{CodeNoLabelTiny}
  \end{block}
  gawk, mawk and nbawk
  \begin{block}{}
    \begin{CodeNoLabelTiny}
      awk '/pattern/'
    \end{CodeNoLabelTiny}
  \end{block}
  \break\break
  Perl
  \begin{block}{}
    \begin{CodeNoLabelTiny}
      perl -ne 'print if /pattern/'
    \end{CodeNoLabelTiny}
  \end{block}
  Ruby
  \begin{block}{}
    \begin{CodeNoLabelTiny}
      ruby -ne 'puts $_ if /pattern/'
    \end{CodeNoLabelTiny}
  \end{block}
\end{frame}

%%%%%%%%%%%%%%%%%%%%%%%%%%%%%%%%%%%%%%%%%%%%%%%%%%%%%%%%%%%%%%%%%%%%%%
\tiny
\begin{frame}
  \frametitle{Benchmark. Results} Cells contain time in nanoseconds
  required for processing one symbol. Time limit is 1000 ns per
  symbol. Columns correspond to different average length of the
  scanned lines. Gray raw contains an average text length to match.
\begin{table}
   \begin{tabular}{ l | c |  c |  c |  c |  c |  c |  c |  c |  c | }
   \rowcolor{lightgray} & 8.95 & 17.90 & 35.80 & 71.60 & 143.15 & 286.12 & 572.25 & 1145.43 & 2290.71 \\
   \hline
   \cellcolor{lightgray} my\_grep & 3.61 & 2.85 & 2.51 & 2.29 & 2.22 & 2.20 & 2.15 & 2.16 & 2.18 \\
   \cellcolor{lightgray} GNU libc & 11.11 & 8.74 & 9.03 & 12.28 & 19.92 & 37.11 & 71.62 & 140.94 & 277.00 \\
   \cellcolor{lightgray} NetBSD libc & 78.51 & 71.96 & 70.09 & 69.12 & 67.78 & 67.73 & 67.66 & 67.60 & 67.37 \\
   \cellcolor{lightgray} libtre & 17.50 & 15.14 & 14.58 & 13.93 & 13.54 & 13.32 & 13.22 & 13.23 & 13.01 \\
   \cellcolor{lightgray} libpcre2 & 7.23 & 6.04 & 7.09 & 11.78 & 26.69 & 71.05 & 219.58 & 746.37 & --- \\
   \cellcolor{lightgray} liboniguruma & 14.99 & 11.65 & 12.96 & 21.95 & 63.79 & 297.57 & --- & --- & --- \\
   \cellcolor{lightgray} libuxre & 4.06 & 3.12 & 2.63 & 2.45 & 2.33 & 2.27 & 2.20 & 2.20 & 2.15 \\
   \cellcolor{lightgray} librxspencer & 37.05 & 33.37 & 32.10 & 30.19 & 29.26 & 30.40 & 29.25 & 30.45 & 29.35 \\
   \cellcolor{lightgray} std::regex & 29.89 & 35.66 & 61.66 & 146.04 & 515.53 & --- & --- & --- & --- \\
   \cellcolor{lightgray} re2 & 10.72 & 6.07 & 3.64 & 2.26 & 1.59 & 1.20 & 1.00 & 0.90 & 0.84 \\
   \cellcolor{lightgray} PIRE & 4.21 & 2.48 & 1.41 & 0.77 & 0.41 & 0.27 & 0.18 & 0.13 & 0.11 \\
   \cellcolor{lightgray} GNU grep & 2.53 & 2.07 & 1.88 & 1.71 & 1.62 & 1.52 & 1.47 & 1.47 & 1.41 \\
   \cellcolor{lightgray} Perl & 12.89 & 8.70 & 7.60 & 12.97 & 44.87 & 232.52 & --- & --- & --- \\
   \cellcolor{lightgray} Ruby & 40.33 & 26.08 & 21.24 & 25.28 & 62.33 & 276.20 & --- & --- & --- \\
   \cellcolor{lightgray} GNU awk & 11.50 & 7.34 & 4.86 & 3.42 & 2.62 & 2.19 & 2.07 & 1.89 & 1.82 \\
   \cellcolor{lightgray} mawk & 5.14 & 3.36 & 2.82 & 3.04 & 5.39 & 14.93 & 50.88 & 196.76 & 837.55 \\
   \cellcolor{lightgray} nbawk & 15.27 & 10.16 & 7.69 & 6.37 & 5.63 & 5.23 & 5.10 & 5.00 & 4.93 \\
   \end{tabular}
   \caption{Performance table. The lower numbers are -- the better}
\end{table}
\end{frame}
\normalsize

%%%%%%%%%%%%%%%%%%%%%%%%%%%%%%%%%%%%%%%%%%%%%%%%%%%%%%%%%%%%%%%%%%%%%%
\footnotesize
\begin{frame}[fragile]
  \frametitle{Conclusion}
  \begin{itemize}
  \item The following libraries and tools are clean loosers: gcc
    std::regex (C++), libpcre2,
    liboniguruma, Perl and Ruby. So, on
    large texts DO NOT USE THEM!
  \item Something bad happens with nbawk, mawk. They use DFA internally, but
    probably work in badly implemented ``substring search'' mode.
    We need more investigation. GNU libc also needs further analysis.
  \item Something strange happens with NetBSD grep. It also requires
    more investigation.
  \item Yandex PIRE, Google re2, libuxre, libtre, my\_grep, GNU
    grep and GNU
    AWK demonstrate linear complexity of algorithm and therefore work
    very well.
  \item At least on this particular workload Yandex PIRE and Google re2
    demonstrates the best performance.
  \item Performance of my\_grep and libuxre are almost the same. This probably
    means that algorithms are the same. The same concerns mawk and
    nbawk.
  \item librxspencer and NetBSD libc --- no comments yet.
  \end{itemize}
\end{frame}
\normalsize

%%%%%%%%%%%%%%%%%%%%%%%%%%%%%%%%%%%%%%%%%%%%%%%%%%%%%%%%%%%%%%%%%%%%%%
\smaller
\begin{frame}[fragile]
  \frametitle{Analysis}
  \break
  \textbf{Note:} From re2 library:
  \begin{itemize}
  \item \textbackslash{}1 --- backreference NOT SUPPORTED
  \item \textbackslash{}g1 --- backreference NOT SUPPORTED
  \item \textbackslash{}g{1} --- backreference NOT SUPPORTED
  \item \textbackslash{}g{+1} --- backreference NOT SUPPORTED
  \item \textbackslash{}g{-1} --- backreference NOT SUPPORTED
  \item \textbackslash{}g{name}	--- named backreference NOT SUPPORTED
  \item \textbackslash{}k$<${}name$>$ --- named backreference NOT SUPPORTED
  \item \textbackslash{}named backreference NOT SUPPORTED
  \end{itemize}
  \break
  \textbf{My personal recomendation:} If you see that regular
  expression library supports backreferences, think twice
  before using it for working with large texts.
\end{frame}
\normalsize

%%%%%%%%%%%%%%%%%%%%%%%%%%%%%%%%%%%%%%%%%%%%%%%%%%%%%%%%%%%%%%%%%%%%%%
\begin{frame}[fragile]
  \frametitle{}
  \Huge{to be continued...}
\end{frame}

%%%%%%%%%%%%%%%%%%%%%%%%%%%%%%%%%%%%%%%%%%%%%%%%%%%%%%%%%%%%%%%%%%%%%%
\end{document}
